%---------------------------------导言------------------------------------
%$\documentclass{ctexart}
\documentclass {ctexart}
\setCJKmainfont[ItalicFont={AR PL UKai CN}]{AR PL UMing CN} %设置中文默认字体
\setCJKsansfont{WenQuanYi Zen Hei} %设置文泉驿正黑字体作为中文无衬线字体
\setCJKmonofont{WenQuanYi Zen Hei Mono} %设置文泉驿等宽正黑字体作为中文打字机字体
\usepackage{amsmath,amsthm,amsfonts,amssymb,bm} % 数学宏包
\usepackage{fancyhdr} %页眉设置
\usepackage[Lenny]{fncychap} %章节样式
\usepackage{color} % 支持彩色
\usepackage{pifont} % 数字加圈 
\usepackage{multicol}
\usepackage{multirow}


\usepackage[colorlinks,unicode,urlcolor=blue,linkcolor=blue,
        pdftitle=学员-UserName合同,pdfauthor=ProcessOne,pdfsubject=leodeu,pdfkeywords=leoedu, pdfpagelabels=false]{hyperref}

% == 设置页面
\usepackage[top=3cm,left=2cm,right=2cm,bottom=2cm]{geometry}
\geometry{papersize={21cm,29.7cm}}
%--------------------------------结束-------------------------------------
%-------------------------------导言--------------------------------------
\newsavebox{\headpic}
\sbox{\headpic}{\includegraphics[height=1.2cm]{/var/www/admin.yb1v1.com/app/OrderPdf/1.png}} %设置页眉logo页眉


\newcommand{\yihao}{\fontsize{34pt}{\baselineskip}\selectfont} % 一号字体
\newcommand{\sanhao}{\fontsize{18pt}{\baselineskip}\selectfont} % 三号字体

\usepackage[rgb]{xcolor}
\definecolor{OliveGreenRGB}{rgb}{0.0,0.6,0.0}

\usepackage{draftwatermark}
\SetWatermarkAngle{0}
\SetWatermarkColor{black!60!cyan!100}
\SetWatermarkFontSize{2cm}
%\SetWatermarkText{理优教育}
\SetWatermarkText{\includegraphics{/var/www/admin.yb1v1.com/app/OrderPdf/b1.png}}
\usepackage{fontspec}

\usepackage{makeidx}

\renewcommand\thesubsection{ \textcolor{OliveGreenRGB}{   }  }
%------------------------------结束--------------------------------------

%-----------------------------开始正文-----------------------------------



\makeindex
\begin{document}


\setlength{\parskip}{0.7ex plus0.3ex minus0.3ex} % 段落间距

\setcounter{page}{1} %这页开始设置为计数
\headheight 14pt % 页眉高


\pagestyle{fancy}

\fancyhf{}
\renewcommand{\headrulewidth}{0pt}
%\fancyhead[LE,RO]{ \usebox{\headpic} \textbf{\thepage} }      %Displays the page number in bold in the header,
\fancyhead[LE,RO]{ \usebox{\headpic}   }      %Displays the page number in bold in the header,
                                           % to the left on even pages and to the right on odd pages.
%\fancyhead[RE]{\nouppercase{\leftmark}}    %Displays the upper-level (chapter) information---
                                           % as determined above---in non-upper case in the header, to the right on even pages.
%\fancyhead[LO]{\rightmark}                 %Displays the lower-level (section) information---as
                                           % determined above---in the header, to the left on odd pages.

%调用页眉:rhead是logo图标放在右上角,左上角为lhead,中间位chead
\subsection{}


\begin{table}[tp]  
  
  \centering  
  \label{tab:performance_comparison}  
  \begin{tabular}{|c|c|c|c|c|c|c|}  
    \hline  
    \multirow{2}{2cm}{Method}&  
    \multicolumn{3}{c|}{C}&\multicolumn{3}{c|}{ D}\cr\cline{2-7}  
    &Precision&Recall&F1-Measure&Precision&Recall&F1-Measure\cr  
    \hline  
    A&0.7324&0.7388&0.7301&0.6371&0.6462&0.6568\cr\hline  
    B&0.7321&0.7385&0.7323&0.6363&0.6462&0.6559\cr\hline  
    C&0.7321&0.7222&0.7311&0.6243&0.6227&0.6570\cr\hline  
    D&0.7654&0.7716&0.7699&0.6695&0.6684&0.6642\cr\hline  
    E&0.7435&0.7317&0.7343&0.6386&0.6488&0.6435\cr\hline  
    F&0.7667&0.7644&0.7646&0.6609&0.6687&0.6574\cr\hline  
    G&{\bf 0.8189}&{\bf 0.8139}&{\bf 0.8146}&{\bf 0.6971}&{\bf 0.6904}&{\bf 0.6935}\cr  
    \hline  
  \end{tabular}  
\end{table}  

\begin{table}
  \centering  
  \begin{tabular}{|c|c|c|c|c|}
    \hline
    贷款方&分期期数&手续费及利息费率&退款违约金&甲方贴息费率\cr\hline
    \multirow{4}{*}{小牛分期} &  3期  & 4.80\% &  - & - \\
    \cline{2-5}   %  \cline用于画横线 \cline{i-j}表示从第i列画到第j列
    &  6期  & 4.80\% &  - & - \\
    &  9期  & 14.40\% &  - & - \\
    &  12期  & 19.20\% &  - & - \\

    \multirow{2}{*}{小牛分期} &
    \multicolumn{2}{|c|}{AA } &
    \multicolumn{2}{|c|}{\multirow{2}{*}{BB}} \\
    \cline{2-3}   %  \cline用于画横线 \cline{i-j}表示从第i列画到第j列
    & column-1 & column-2 & \multicolumn{2}{|c|}{KKKK} \\
    \hline
    label-1 & label-2 & label-3 & label-4 & label-5 \\
    \hline
  \end{tabular}
\end{table}

\printindex

\end{document}
